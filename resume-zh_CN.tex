% !TEX TS-program = xelatex
% !TEX encoding = UTF-8 Unicode
% !Mode:: "TeX:UTF-8"

\documentclass{resume}
\usepackage{zh_CN-Adobefonts_external} % Simplified Chinese Support using external fonts (./fonts/zh_CN-Adobe/)
%\usepackage{zh_CN-Adobefonts_internal} % Simplified Chinese Support using system fonts
\usepackage{linespacing_fix} % disable extra space before next section
\usepackage{cite}
\usepackage{hyperref}
\usepackage{tabularx}

\begin{document}
\pagenumbering{gobble} % suppress displaying page number

\name{翟召轩}

% {E-mail}{mobilephone}{homepage}
% be careful of _ in emaill address
\contactInfo
{(+86) 17610609826}
{kxuanobj@gmail.com}
{\href{https://www.linkedin.com/in/\%E5\%8F\%AC\%E8\%BD\%A9-\%E7\%BF\%9F-29463a115/}{LinkedIn}}
{\href{https://github.com/kxuan}{GitHub @Kxuan}}{}

\section{个人总结}
{
资深嵌入式系统/软件工程师,
拥有9年+在Canonical (Ubuntu)、OPPO、小米、美的等知名企业的工作经验。
专注于操作系统移植(Ubuntu)、嵌入式软件架构设计、IoT协议开发(Matter, HeyThings)及网络安全。
主导多项核心技术落地,拥有30余项发明专利。
熟练掌握C/Rust/Python,精通构建系统(Meson/CMake)及嵌入式系统设计。追求高性能、高可靠性的技术解决方案。
}

\section{工作经历}
\datedsubsection{\textbf{Canonical - 科能(上海)软件科技有限公司}, Software Engineer II,远程}{2023.09-至今}
\begin{itemize}
  \item 负责将 公司核心产品 \textbf{Ubuntu 操作系统}移植并适配到 MediaTek Genio 平台
  \item 负责硬件特色功能在用户空间(User Space)的适配,包括APU (AI Processing Unit)、Camera、Codec、GPU等硬件功能
\end{itemize}

\datedsubsection{\textbf{美的 - 广东美的制冷设备有限公司},嵌入式研发资深工程师,深圳}{2022.11-2023.02}
\begin{itemize}
  \item 主导美的智能家居端侧中枢网关功能的开发,目标是实现对所有事业部智能家居产品通信协议及 Matter 协议的统一接入与整合
\end{itemize}

\datedsubsection{\textbf{OPPO - OPPO广东移动通信有限公司},高级系统工程师,深圳}{2019.05-2022.10}
\begin{itemize}
  \item 主导/参与 \textbf{发明专利 30余项} (\href{https://patents.google.com/?inventor=\%E7\%BF\%9F\%E5\%8F\%AC\%E8\%BD\%A9\&oq=\%E7\%BF\%9F\%E5\%8F\%AC\%E8\%BD\%A9}{部分公开授权专利 [Google Patents链接]})
  \item 主导 HeyThings IoT 网络通信协议的设计与落地;分阶段推进在 照明、摄像头设备、扫地机器人落地
  \item 负责 HeyThings RTOS SDK \textbf{软件架构}、\textbf{构建系统}。实现SDK基础框架功能(包括网络、事件系统、硬件适配层等),作为 HeyThings 协议的参考实现。
\end{itemize}

\datedsubsection{\textbf{小米 - 北京小米移动软件有限公司},高级系统工程师,北京}{2017.05-2019.05}
\begin{itemize}
  \item 负责 米家多功能网关产品 \textbf{规则引擎} 功能
  \item 负责 米家摄像头品类 设备侧SDK \textbf{架构更新},并负责基础子系统开发设计(包括音视频流、存储等)
  \begin{itemize}
    \item 新架构较第一代架构节省 50\% ROM 空间需求(32MB → 16MB)
    \item 新架构高效稳定可靠,代码已通过\textbf{数亿台设备}的量产验证
  \end{itemize}
\end{itemize}

\datedsubsection{\textbf{Fujitsu - 南京富士通•南大软件技术有限公司},软件工程师,南京}{2016.02-2017.05}
\begin{itemize}
  \item 针对富士通服务器的硬件性能进行性能测试及针对性调优。为客户提供系统性能配置建议。
  \item 参与华为私有云项目,负责 Guest OS 问题验证、排查故障原因及提供修复补丁
\end{itemize}

\section{技能}
\begin{itemize}
  \item \textbf{操作系统}: 深入理解 Linux 系统原理,Ubuntu 系统开发经验
  \item \textbf{网络与安全}: 精通计算机网络原理及常见互联网协议;熟练应用密码学技术保障网络通信安全
  \item \textbf{嵌入式系统}: 精通嵌入式软件架构设计、硬件适配、性能优化
  \item \textbf{编程语言}: 精通 C;熟练掌握 Rust, Python, JavaScript
  \item \textbf{构建系统}: 精通 Meson, CMake, Makefile,擅长搭建跨平台构建系统
  \item \textbf{语言}: 中文(母语);英语(工作流利,TOEIC L\&R 850)
\end{itemize}


\section{奖项与认可}
\begin{itemize}
  \item Canonical 内部竞赛 Madrid CTF 第一名;TheHague CTF 第三名;
  \item OPPO 三年(2019-2021)年度绩效均为A;
  \item OPPO 2022年软件工程系统(含印度、美国分部)编程竞赛(算法类)月赛排名最高第12名;
  \item 2017年 小米 MIoT 部门最佳新人奖;
  \item 2016年 南京富士通•南大软件技术有限公司 最佳新人奖;
  \item 2016年 中科院软件研究所“卓越工程师”培养项目特等奖学金;
\end{itemize}

% \section{\faGraduationCap\ 教育背景}
\section{教育背景}
\datedsubsection{\href{https://www.jit.edu.cn/}{\textbf{金陵科技学院}},软件工程学院,软件工程专业,\textit{工学学士学位}}{2012-2016}

\renewcommand{\thefootnote}{}
\footnotetext{Updated on \today}

%% Reference
%\newpage
%\bibliographystyle{IEEETran}
%\bibliography{mycite}
\end{document}
